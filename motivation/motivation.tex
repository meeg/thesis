\chapter{Motivation}
Massive U(1) vector bosons, also known as heavy photons, are a natural consequence of many theories of physics beyond the Standard Model.
Such a particle could kinetically mix with the photon, giving it an effective coupling to electric charge much smaller than the photon's direct coupling $\alpha$.
A heavy photon is one of several ``portals'' by which a dark sector could interact with Standard Model matter.

The existence of such a heavy photon is a possible explanation for the cosmic ray positron excess and the muon $g-2$ anomaly.

\section{Theory}
The $A'$ can be massive or massless.


Holdom \cite{holdom_two_1986} summarizes the implications.

The kinetic mixing means that the two $U(1)$ fields are not orthogonal.
If we diagonalize

If the new $U(1)$ is massless, the photon can be
If the $A'$ is massless, the new $U(1)$ mixes 
particles charged under the $A'$ gain a small electromagnetic charge. 

Such millicharged particles have been the subject of direct experimental searches


If the $A'$ is massive, particles charged under the $A'$ do not 

massless U(1)': millicharge, paraphoton
paraphoton doesn't couple to normal matter, para-matter gets millicharged

massive U(1)': heavy photon
photon doesn't couple to hidden sector, heavy photon couples to SM matter


Snowmass \cite{essig_dark_2013}

millicharge \cite{davidson_updated_2000}

one-loop coupling: fermions with both charges



\section{Signatures}

production cross-section

decay length

\section{Other Searches}


colliders

meson decays (LHCb)

positron on fixed target (VEPP)

old beam dumps

SeaQuest
