\chapter{Motivation}
Massive U(1) vector bosons, also known as heavy photons, are a natural consequence of many theories of physics beyond the Standard Model.
Such a particle could kinetically mix with the photon, giving it an effective coupling to electric charge much smaller than the photon's direct coupling $\alpha$.
A heavy photon is one of several ``portals'' by which a dark sector could interact with Standard Model matter.

The existence of such a heavy photon is a possible explanation for the cosmic ray positron excess and the muon $g-2$ anomaly.

\section{Theory}


\section{Signatures}

production cross-section

decay length

\section{Other Searches}


colliders

meson decays (LHCb)

positron on fixed target (VEPP)

old beam dumps

SeaQuest
