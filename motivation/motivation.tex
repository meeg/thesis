\chapter{Motivation}
There is current theoretical and experimental interest in a massive $U(1)$ vector boson that does not couple directly to particles in the Standard Model, but gains a weak effective coupling to charged particles through kinetic mixing.
This particle is commonly called a heavy photon, dark photon, or $A'$, and is characterized by its mass $m_{A'}$ and a dimensionless coupling constant $\epsilon$.

The heavy photon is primarily motivated as part of a larger ``dark sector'' of particles that are not charged directly under the Standard Model forces.
Some sort of ``portal'' is necessary to create an interaction between the dark sector and the Standard Model.
The possible portals are restricted by the symmetries of the Standard Model, and the dominant candidates are commonly referred to as the vector, Higgs, neutrino, and axion portals; the heavy photon is the mediator for the vector portal \cite{essig_dark_2013}.
Dark sector particles are natural candidates for dark matter, and a heavy photon can be part of a mechanism that produces the observed dark matter abundance.

%Massive U(1) vector bosons, also known as heavy photons, are a natural consequence of many theories of physics beyond the Standard Model.
%Such a particle could kinetically mix with the photon, giving it an effective coupling to electric charge much smaller than the photon's direct coupling $\alpha$.
%A heavy photon is one of several ``portals'' by which a dark sector could interact with Standard Model matter.
%
%The existence of such a heavy photon is a possible explanation for the cosmic ray positron excess and the muon $g-2$ anomaly.

\section{Theory Summary}
The basic assumption behind the heavy photon is that there exists a second (broken) $U(1)$ symmetry, and that it interacts with Standard Model hypercharge via kinetic mixing \cite{holdom_two_1986}.
At low energies, this leads to the following gauge field Lagrangian, where $F_{\mu\nu}=\partial_\mu A_\nu - \partial_\nu A_\mu$ is the electromagnetic field strength, $F'_{\mu\nu} = \partial_\mu A'_\nu - \partial_\nu A'_\mu$ is the heavy photon field strength, and $\epsilon$ is the dimensionless coupling constant:

\begin{equation}
    \mathcal{L}_{\mathrm{gauge}}=-\frac{1}{4}F^{\mu\nu}F_{\mu\nu} - \frac{1}{4}F'^{\mu\nu}F'_{\mu\nu} + \frac{1}{2}\epsilon F^{\mu\nu}F'_{\mu\nu}
\end{equation}

The kinetic mixing means that the fields are not orthogonal.
Orthogonality can be restored by redefining the electromagnetic field according to $A^\mu \to A^\mu + \epsilon A'^\mu$.
This modifies the interaction Lagrangian as follows:

\begin{equation}
    A^\mu J^{EM}_\mu \to A^\mu J^{EM}_\mu + \epsilon A'^\mu J^{EM}_\mu 
\end{equation}

This implies that particles with electric charge acquire a coupling, proportional to $\epsilon$, to the heavy photon.
The converse is not true: hidden-sector particles with heavy photon couplings do not acquire couplings to the photon (such ``millicharged'' particles do occur if the new $U(1)$ is unbroken \cite{prinz_search_1998}) \cite{holdom_two_1986}.

If particles exist that are charged under both fields, kinetic mixing may arise from a one-loop diagram similar to Figure \ref{fig:oneloop}, with a natural scale of $\epsilon \sim 10^{-2}-10^{-4}$; on the other hand, GUT models require that one-loop contributions to $\epsilon$ vanish, and instead motivate two-loop contributions at $\epsilon \sim 10^{-3}-10^{-6}$ \cite{arkani-hamed_lhc_2008}.
Because kinetic mixing is a renormalizable interaction, $\epsilon$ is independent of the masses of the particles that give rise to it.
String theory models can motivate much smaller $\epsilon$, as low as $10^{-12}$ \cite{goodsell_naturally_2009,cicoli_testing_2011}.

\begin{figure}[ht]
    \begin{center}
        \begin{fmffile}{oneloop}
            \begin{fmfgraph*}(150,150)
                \fmfstraight 
                \fmfleft{i1}
                \fmfright{o1}
                \fmflabel{$\gamma$}{i1}
                \fmflabel{$A'$}{o1}
                \fmf{photon,tension=2}{i1,v1}
                \fmf{photon,tension=2}{v2,o1}
                \fmf{fermion,left}{v1,v2}
                \fmf{fermion,left}{v2,v1}
            \end{fmfgraph*}
        \end{fmffile}
    \end{center}
    \caption{One-loop diagram leading to kinetic mixing.}
    \label{fig:oneloop}
\end{figure}

There is a wide range of reasonable values for the mass $m_{A'}$.
Models where supersymmetry breaking is communicated by the kinetic mixing lead to natural mass scales of MeV-GeV \cite{baumgart_non-abelian_2009, morrissey_abelian_2009, cheung_kinetic_2009}.
String theory models typically tie the mass scale to $\epsilon$, and can motivate masses down to the meV scale \cite{goodsell_naturally_2009,cicoli_testing_2011}.

If the new $U(1)$ is massless, the photon can be
If the $A'$ is massless, the new $U(1)$ mixes 
particles charged under the $A'$ gain a small electromagnetic charge. 

Such millicharged particles have been the subject of direct experimental searches


If the $A'$ is massive, particles charged under the $A'$ do not 

massless U(1)': millicharge, paraphoton
paraphoton doesn't couple to normal matter, para-matter gets millicharged

massive U(1)': heavy photon
photon doesn't couple to hidden sector, heavy photon couples to SM matter


Snowmass \cite{essig_dark_2013}

millicharge \cite{davidson_updated_2000}

one-loop coupling: fermions with both charges

\begin{figure}[ht]
    \hspace{5mm}
    \begin{fmffile}{rad1}
        \begin{fmfgraph*}(150,150)
            \fmfstraight 
            \fmfleft{i1,i2,i3,i4}
            \fmfright{o1,o2,o3,o4}
            \fmflabel{$Z$}{i1}
            \fmflabel{$Z$}{o1}
            \fmflabel{$e^-$}{i3}
            \fmflabel{$e^-$}{o2}
            \fmflabel{$e^+$}{o3}
            \fmflabel{$e^-$}{o4}
            \fmf{fermion}{i3,v1,v3,o2}
            \fmf{heavy}{i1,v2,o1}
            \fmf{photon,tension=0,label=$\gamma$}{v1,v2}
            \fmf{fermion}{o3,v4,o4}
            \fmf{photon,tension=2,label=$\gamma$}{v3,v4}
        \end{fmfgraph*}
    \end{fmffile}
    \hspace{10mm}
    \begin{fmffile}{rad2}
        \begin{fmfgraph*}(150,150)
            \fmfstraight 
            \fmfleft{i1,i2,i3,i4}
            \fmfright{o1,o2,o3,o4}
            \fmflabel{$Z$}{i1}
            \fmflabel{$Z$}{o1}
            \fmflabel{$e^-$}{i3}
            \fmflabel{$e^-$}{o2}
            \fmflabel{$e^+$}{o3}
            \fmflabel{$e^-$}{o4}
            \fmf{fermion}{i3,v1,v3,o2}
            \fmf{heavy}{i1,v2,o1}
            \fmf{fermion}{o3,v4,o4}
            \fmf{photon,tension=2,label=$\gamma$}{v1,v4}
            \fmffreeze
            \fmf{photon,tension=0,label=$\gamma$}{v3,v2}
        \end{fmfgraph*}
    \end{fmffile}
    \hspace{5mm}
    \caption{Trident diagrams.}
    \label{fig:tridents_rad}
\end{figure}

\begin{figure}[ht]
    \hspace{5mm}
    \begin{fmffile}{bh1}
        \begin{fmfgraph*}(150,150)
            \fmfstraight 
            \fmfleft{i1,i2,i3,i4}
            \fmfright{o1,o2,o3,o4}
            \fmflabel{$Z$}{i1}
            \fmflabel{$Z$}{o1}
            \fmflabel{$e^-$}{i4}
            \fmflabel{$e^-$}{o2}
            \fmflabel{$e^+$}{o3}
            \fmflabel{$e^-$}{o4}
            \fmf{fermion}{i4,v1,o4}
            \fmf{heavy}{i1,v2,o1}
            \fmffreeze
            \fmf{photon,tension=1,label=$\gamma$}{v1,v3}
            \fmf{photon,tension=1,label=$\gamma$}{v2,v4}
            \fmf{fermion}{o2,v4,v3,o3}
        \end{fmfgraph*}
    \end{fmffile}
    \hspace{10mm}
    \begin{fmffile}{bh2}
        \begin{fmfgraph*}(150,150)
            \fmfstraight 
            \fmfleft{i1,i2,i3,i4}
            \fmfright{o1,o2,o3,o4}
            \fmflabel{$Z$}{i1}
            \fmflabel{$Z$}{o1}
            \fmflabel{$e^-$}{i4}
            \fmflabel{$e^-$}{o2}
            \fmflabel{$e^+$}{o3}
            \fmflabel{$e^-$}{o4}
            \fmf{fermion}{i4,v1,o4}
            \fmf{heavy}{i1,v2,o1}
            \fmffreeze
            \fmf{photon,tension=2,label=$\gamma$}{v1,v3}
            \fmf{photon,tension=2,label=$\gamma$}{v2,v4}
            \fmf{fermion}{o2,v3,v4,o3}
        \end{fmfgraph*}
    \end{fmffile}
    \hspace{5mm}
    \caption{Trident diagrams.}
    \label{fig:tridents_bh}
\end{figure}


\section{Signatures}

production cross-section

decay length

\section{Other Searches}


colliders

meson decays (LHCb)

positron on fixed target (VEPP)

old beam dumps

SeaQuest
