\chapter{Conclusion}
This dissertation covers the initial stages of the HPS experiment.
The detector has been built, and brought into operation.
We are approaching a full understanding of the event reconstruction and the data.
The analyses are approaching their final state.

The displaced-vertex search is fully developed.
The event selection has been tuned, and the backgrounds after event selection have been characterized.
The steps of the analysis have been tested, and perform as expected.
Data and Monte Carlo are each used where appropriate, and the analysis is data-driven to the extent possible.

The remaining steps to a complete result are clear.
The excess background at large $z$ (Section \ref{sec:excess_background}) must be understood and rejected.
Pairs with missing layer 1 hits are an additional data set, which require separate tuning of cuts and modeling of signals and backgrounds, and the data sets must then be combined.
Similarly, the data taken with the tracker at 1.5 mm must be combined with the 0.5 mm data used in this dissertation.

The signal significance analysis detailed in Section \ref{sec:significance} can be refined considerably.
A profile likelihood analysis should make better use of the mass and $z$ information, will directly compute the likelihood of an observed excess being a background fluctuation or a heavy photon, and will identify preferred parameters ($m_{A'}$, $\epsilon^2$, and production rate) for an excess.

This dissertation is written at the time that HPS is becoming a mature experiment.
We now understand how the experiment works, and what we can do to improve future runs.
Based on these results, HPS is planning to reconfigure the SVT to get better acceptance at large z, and an upgrade is being considered to add a ``layer 0'' to improve the vertex resolution.
Reach estimates for future runs will also be much better informed.
