\prefacesection{Abstract}
%Massive U(1) vector bosons, also known as heavy photons, are a natural consequence of many theories of physics beyond the Standard Model.
%Such a particle could kinetically mix with the photon, giving it an effective coupling to electric charge much smaller than the photon's direct coupling $\alpha$.
%A heavy photon is one of several ``portals'' by which a dark sector 
%The existence of such a heavy photon is a possible explanation for the cosmic ray positron excess and the muon $g-2$ anomaly.
%Several current and proposed experiments are dedicated to the search for a heavy photon.

%The Heavy Photon Search (HPS) is a new experiment at Jefferson Lab that will search for a heavy photon with mass 20--1000 MeV and relative coupling $\alpha'/\alpha$ of $10^{-5}-10^{-10}$. 

%The HPS experiment is designed to produce heavy photons by sending the electron beam of Jefferson Lab's CEBAF accelerator through a fixed target.
%The heavy photons will decay to lepton pairs, possibly with some finite decay length; the HPS detector measures the momentum and decay vertex of the pairs.
%The invariant mass and vertex of the pairs are used to identify heavy photons using two signatures (mass resonance and displaced decay vertex).
%The detector is a compact, large-acceptance forward spectrometer comprising a silicon microstrip tracker for momentum measurement and vertexing and an electromagnetic calorimeter for triggering.

%This thesis will describe the physics of and motivations for heavy photons, review the HPS experiment, and concentrate on an analysis of the HPS data to search for heavy photons decaying into $e^+e^-$ pairs.
%Two analyses are planned: a simple ``bump hunt'' looking for a mass resonance, and a vertexing search taking into account both invariant mass and vertex separation.

%The main physics result of this thesis will be an $A'\to e^+e^-$ analysis using the vertexing search.
%This thesis will use data collected in the HPS commissioning run, scheduled for Fall 2014.
%Data from the physics run in 2015 will be used only if commissioning run data is insufficient for a meaningful physics result.

