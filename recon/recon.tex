\chapter{Event Reconstruction and Selection}
\section{Reconstruction}
\subsection{Tracking}
\label{sec:track_recon}

\subsubsection{Hit Reconstruction}
\label{sec:svt_hit_recon}

\subsubsection{Track Finding and Refit}
\label{sec:track_finding_refit}
track finding

GBL refit

\subsection{Vertexing}
\label{sec:vertex_recon}
Pairs of tracks are vertexed using a fast vertex fit that finds the best-fit vertex position and track parameters based on the track parameters and covariance matrices, and optional additional vertex constraints \cite{billoir_fast_1992}.

The HPS vertex reconstruction uses constraints on the $x$, $y$, and $z$ location of the vertex.
All constraints are limited by the vertex resolutions in those directions; the $x$ and $y$ constraints are limited by the beamspot size, and the $z$ constraint is limited by knowledge of the target position, but these are all smaller than the vertex resolutions.
%Three types of constraints are used, all using knowledge about the target and/or beamspot.
The ``$z$-constrained'' fit requires that the vertex be consistent with the $z$ location of the target.
The ``target-constrained'' fit requires that the vertex be consistent with the $z$ location of the target, and with the $x$ and $y$ location of the beam spot.
The ``beamspot-constrained'' fit requires that the vertex position and momentum are such that the vertex momentum points back to the beamspot at the target $z$.

\subsection{Clustering}
\label{sec:clustering}

\subsection{Track-Cluster Matching}
\label{sec:matching}
\section{Tracker Performance and Alignment}
\subsection{Internal Alignment}
millepede
\subsection{Elastic Electrons}
\label{sec:target_z}
\subsection{M{\o}ller Electrons}
\label{sec:mollers}

\section{\texorpdfstring{$e^+e^-$}{e+e-} Selection Cuts}
\label{sec:event_selection}
After reconstruction, all possible $e^+e^-$ pairs in the event are tested against a set of cuts.
There is no explicit requirement that the electron and positron be the pair of particles that caused the trigger.
There is also no fiducial requirement, since the inner edge of the detector acceptance is key for sensitivity to low-mass heavy photons.
The base selection is intended to remove accidental coincidences from the pair sample.

The ``pairs-1'' trigger is the HPS physics trigger, described in Section \ref{sec:trigger_cuts}.
It is tuned to accept $e^+e^-$ pairs, and is the overwhelming majority of the event rate (16.6 kHz out of 19 kHz).

Track-cluster matching is important for two reasons: the cluster time resolution is better than the track time resolution, and track-cluster matching eliminates many misreconstructed tracks.
Two checks are done on the quality of the track-cluster matching for both particles.
First, a cut is made on the $\chi^2$ of the track-cluster match; this is a requirement on the distance between the cluster position and the track extrapolation to the ECal.
Second, a cut is made on the track-cluster time difference.
Since the track and cluster times are referenced differently (the track time is relative to the trigger time, and the cluster time is relative to the start of the ECal readout window), a constant offset of 43 ns is subtracted.

The electron and positron are required to be in opposite halves of the detector: this cut is implemented as a requirement that the $y$-coordinates of the two clusters have opposite signs.
The trigger requires a top-bottom coincidence, so repeating the requirement as an event selection cut does not reduce the efficiency.
This avoids any possibility of confusion in the track or cluster reconstruction, since the hits from the electron and positron are guaranteed to be well separated.
%A heavy photon can have enough transverse momentum that both decay products land in the same half of the detector, but the rate is low.

Three more simple cleanup cuts are applied.
A loose track quality cut is applied on the $\chi^2$ of each GBL fit; this is only meant to reject very poor track fits.
Elastically scattered electrons with $p(e^-)\approx E_{beam}$ are the main pileup background in the tracker, and are rejected with a maximum momentum requirement on electrons.
A momentum sum cut rejects pairs with a momentum sum too far in excess of $E_{beam}$; this further reduces the rate of random coincidences with elastic electrons.

Finally, a ``radiative cut'' is applied for heavy photon analyses.
This is a minimum requirement on the momentum sum, at $0.8E_{beam}$.
As shown in Section \ref{sec:signal_kinematics}, most heavy photons and radiative tridents are produced with energy near $E_{beam}$; the radiative cut keeps most of these and rejects the Bethe-Heitler tridents that dominate at low momentum.

\begin{table}[h]
    \begin{center}
        \begin{tabular}{lc}   
            \hline \hline
            Trigger type & ``pairs-1'' trigger \\
            Track-cluster matching (position) & $\chi^2_{match}<10$ \\
            Track-cluster matching (time) & $|t_{cl}-t_{trk}-43|<4$ ns \\
            Cluster time coincidence & $|t_{cl}(e^-)-t_{cl}(e^+)|<2$ ns \\
            Top-bottom requirement & $\sign(y_{cl}(e^-))\neq\sign(y_{cl}(e^-))$ \\
            Track quality & $\chi^2_{trk}<50$ \\
            Elastics cut & $p(e^-)<0.75E_{beam}$ \\
            Momentum sum cut & $p_{tot}(e^+e^-)<1.15E_{beam}$ \\
            Radiative cut & $p_{tot}(e^+e^-)>0.8E_{beam}$ \\
            \hline \hline
        \end{tabular}
        \caption{Base pair selection cuts for HPS.}
        \label{tab:basic_cuts} 
    \end{center}
\end{table}

\subsection{Tuning Cuts}
The pair selection cuts are tuned on the data, using the cluster time difference to separate ``good'' and ``bad'' events.
Pairs with large cluster time difference ($|t_{cl}(e^-)-t_{cl}(e^+)|>3$ ns) are accidental coincidences; pairs with small cluster time difference ($|t_{cl}(e^-)-t_{cl}(e^+)|<1$ ns) are dominated by true time coincidences.
An effective cut should reject pairs with large cluster time difference, and not pairs with small cluster time difference.


\section{Data Processing and Normalization}
\label{sec:luminosity}


\subsection{Rate Comparison with Monte Carlo}
\label{sec:rates}