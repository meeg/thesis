\chapter{Pulse Fitting}
The APV25 readout chip

\section{Pulse Model}
2-pole (CR-RC) model

3-pole model

coincident poles

electronics

\section{APV25 Readout}
The tracker will be made up of six measurement layers, each containing two closely spaced planes of silicon microstrip sensors to measure both X and Y coordinates for momentum measurement and track identification. 
We will read out the microstrips using the APV25 chip developed for the CMS tracker at the LHC. 
The APV25 preamplifier and shaper produce a CR-RC shaping curve with a shaping time $T_p$ which, for our experiment, will be set at 35 ns. 
The approximate functional form of this curve, for hit time of $t_0$ and peak height (proportional to the charge generated in the silicon) of $A$, is:
$$v_{out}(t)=A\frac{t-t_0}{T_p}e^{-(t-t_0)/T_p}$$
If there are multiple hits, the output is the sum of the shaping curves for the individual hits.

The APV25 samples the shaper output once per clock cycle (24 ns for HPS), and in ``multi-peak'' mode, outputs 3 consecutive samples per trigger request. HPS will read out 6 samples for each trigger. The trigger timing can be controlled to select the series of 6 samples that gives us the most information for a fit; currently we believe that it is best to take 2 samples before the signal hit and 4 samples after.

In general, we want to fit a set of samples $y_i$ and errors $\sigma_i$, taken at times $t_i$, with the sum of pulses with shaping curve $f(t)$, each of height $A_j$ and hit time $t_j$.
$$\chi^2 = \sum_i\left(\frac{y_i-\sum_j A_j f(t_i-t_j)}{\sigma_i}\right)^2$$

\section{Background Pileup}
With our high background rate, it is expected that the ``signal'' hit corresponding to the track to be fitted will often overlap with ``background'' hits. 
Overlapping hits sum to make a fictitious ``pileup'' hit, whose apparent height and hit time are weighted sums of the signal and background values. 

%\begin{figure}
%\begin{center}
%\includegraphics[width=\textwidth]{gnuplot/pileup.png}
%\end{center}
%\caption{Effect of pileup on a signal hit at 34 ns and a background hit at -17 ns; the apparent hit time is shifted earlier by 9.6 ns.}
%\end{figure}

There is no way to distinguish a ``pileup'' hit from a pure signal hit, except by fitting the samples taken before the pileup. 
This is only possible if at least two samples are taken before pileup; otherwise the time and height of the earlier hit are underconstrained.

%\begin{figure}
%\begin{center}
%\includegraphics[width=0.8\textwidth]{pileup_underconstrained.png}
%\end{center}
%\caption{Example of an underconstrained fit. Samples (red asterisks) match both the actual hit times and heights (black curve) and the fitted times and heights (red curve); many other fits are possible.}
%\end{figure}


\section{Fitting Methods}

$\chi^2$ is minimized when its partial derivative is 0 with respect to all fit parameters. 
%$$\frac{\partial\chi^2}{\partial A_k} = -2 \sum_i\left(\frac{y_i-\sum_j A_j f(t_i-t_j)}{\sigma_i}\right)\frac{f(t_i-t_k)}{\sigma_i}$$
The condition $\frac{\partial\chi^2}{\partial A_k}=0$ yields the following:
$$\sum_i\left(\frac{\sum_j A_j f(t_i-t_j)}{\sigma_i}\right)\frac{f(t_i-t_k)}{\sigma_i}= \sum_i\left(\frac{y_i}{\sigma_i}\right)\frac{f(t_i-t_k)}{\sigma_i}$$

This system of equations can be solved as a matrix equation to efficiently obtain the best-fit values of $A_j$ given a set of $t_j$; this works generally, even if the shaping curve is not exactly the ideal CR-RC curve. 
The problem is reduced to that of finding best-fit values of $t_j$, and this can be done analytically or numerically.

\section{Analytic Fitting}
For a fully analytic fit, we notice that the samples can be partitioned into sets, each of which can be fitted by a single shaping curve; 
these can be fitted individually, and then pileup can be subtracted to extract the heights and times of all the hits.

%The condition $\frac{\partial\chi^2}{\partial t_k}=0$ yields the following:
%$$\sum_i\left(\frac{\sum_j A_j f(t_i-t_j)}{\sigma_i}\right)\frac{\frac{\mathrm{d}f}{\mathrm{d}t}(t_i-t_k)}{\sigma_i} = \sum_i\left(\frac{y_i}{\sigma_i}\right)\frac{\frac{\mathrm{d}f}{\mathrm{d}t}(t_i-t_k)}{\sigma_i}$$
The individual fits can be made analytically if we assume the ideal CR-RC curve; to find the values of $A$ and $t_0$ to minimize $\chi^2$, we set $\frac{\partial\chi^2}{\partial A}=0$, $\frac{\partial\chi^2}{\partial t_0}=0$.

There are currently two weaknesses to this method. One is that an analytic fit using one set of samples does not know about the other samples; the other is that the analytic fit sometimes yields unphysical results with negative peak heights. Both problems are shown in Figure fig:analytic below. It is expected that these effects can be dealt with as special cases.
%\begin{figure}
%\begin{center}
%\includegraphics[width=0.8\textwidth]{analytic.png}
%\end{center}
%\label{fig:analytic}
%\caption{Attempt at an analytic fit for a two-hit event. An analytic fit to samples 1 and 2 gives an unphysical result, so the best fit is one that fits sample 1 with one peak, and samples 2--6 with the second peak. Because the first peak doesn't try to fit sample 2, sample 2 is ``buried'' by the first peak.}
%\end{figure}


\section{Numeric Fitting}
The values of $t_j$ can be fit by searching the space of possible $t_j$. 
Most gradient-based fitting methods are ineffective for this problem because the global minimum (where all samples are fit) is often at one end of a narrow ``valley'' (where one sample is not fit), but a method that follows this valley should work. Further work is needed.

%\begin{figure}
%\begin{center}
%\includegraphics[width=\textwidth]{linfit_landscape.png}
%\end{center}
%\caption{Contour plot of $\chi^2$ values for a two-hit event, fitted with two peaks. $A_1$ and $A_2$ are fit analytically, given values of $t_1$ and $t_2$ ($x$ and $y$ coordinates, relative to time of first sample). Black asterisks mark the true values of $(t_1,t_2)$; red marks the fitter output. The ``valley'' consists of fit parameters for which only samples 2--6 are fit; the global minimum occurs where sample 1 is fit as well.}
%\end{figure}
%\begin{figure}
%\begin{center}
%\includegraphics[width=0.8\textwidth]{linfit.png}
%\end{center}
%\caption{For the fit above: samples (red asterisks), actual hit parameters (black curve), fitted hit parameters (red curve), and $\chi^2$ dependence on $t_1$ and $t_2$ (green and blue curves).}
%\end{figure}

\section{Time and Height Resolution}
To test performance of the fitters, we generated one-peak and two-peak events and compared the reconstructed values of parameters to the true values. 
For these tests, we assume all hits have the same height (in reality we expect a Landau distribution of heights), with a signal-to-noise ratio of 25; we assume that we can time the trigger so the signal hit is always between samples 2 and 3 (between 24 and 48 ns after the first sample).

%\begin{figure}
%\begin{center}
%\includegraphics[width=0.45\textwidth]{time_diff_2d.png}
%\includegraphics[width=0.45\textwidth]{height_diff.png}
%\end{center}
%\caption{Time and height resolution for single hits. $\sigma_t = 1.37$ ns, $\sigma_A/A = 0.026$.}
%\end{figure}
%\begin{figure}
%\begin{center}
%\includegraphics[width=\textwidth]{2peak_tsig_diff_tdiff.png}
%\end{center}
%\caption{Error in two-peak fit for the signal time. $x$-coordinate is difference between the true hit times; $y$-coordinate is the difference between fitted and true signal hit times.}
%\end{figure}

\section{Identifying Multiple Hits}
To be useful, our fitter must be able to distinguish single-hit events from multiple hits. 
This is not possible if multiple hits fall in the same time interval between samples, but in all other cases we can fit the samples with a single peak and multiple peaks, and compare the goodness of fit.
%\begin{figure}
%\begin{center}
%\includegraphics[width=0.45\textwidth]{prob_tback.png}
%\includegraphics[width=0.45\textwidth]{2peak_prob_tback.png}
%\end{center}
%\caption{$\chi^2$ probabilities for single-peak (left) and two-peak (right) fits to two-peak events. 
    %$x$-coordinate is time of the background hit. 
        %A single-peak fit is good when the background hit and signal hit are both between samples 2 and 3; 
    %the two-peak fit is good when there are two samples for the background hit before pileup, and less good (not enough degrees of freedom) when there is only one.}
    %\end{figure}
